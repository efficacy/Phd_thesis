\subsection{Products and projects}

As well as considering the situation and constraints on the software development process it is important to distinguish between a \emph{product} and a \emph{project}. Unfortunately, these terms can seem largely synonymous, but they have a very different character. This can cause problems when what is needed is a product, but the budget has been decided on the basis that it is a project.

\subsubsection{Products}

A \emph{product} in this context is a software artefact with its own existence. As mentioned above this may be a `shrink-wrap' product sold for installation on a user's computer, it may be an online application or service used by many users, or it may be the embedded software which controls a physical device. The important aspect of this is that the product in some sense has a life of its own which extends beyond that of any particular episode of development work. The exact nature of the relationship between software developers and the product varies according to the commercial and process choices of the organisation. Some products are created and released, then operate largely untended until some major fault is found or some change or new feature is required. Others require continual operational tinkering. Some have a rigid version and release plan, while some change as soon as new code or new ideas are available. Where physical hardware is involved, embedded software can sometimes be updated during use, but other times will need to be returned for refurbishment.

\subsubsection{Projects}

A \emph{project} in this context is the confluence of people, time and resources to achieve a particular objective. A project is a very flexible concept, but this research will concentrate on the notion of a software development project, particularly one which is associated with a particular software (or combined software and hardware) product. A software development project typically has a team of people associated with it, a goal, and a deadline. In most cases there is also the constraint of a budget. A product will typically have several associated projects during its lifetime, including one to create the first version and sometimes one to decommission the product once it reaches the end of its life.

\subsubsection{Single Project Products}

In academic software development working with a fixed funding allocation, the concepts of product and project can often be conflated. In such scenarios the product or service ends at the end of the project, so the two are treated as essentially the same thing. This also applies to student research projects with a hard assessment deadline. The side-effects of this conflation can be seen in software documentation which confuses the end result with the means by which it was achieved. This simplification of concepts can be a useful shortcut, but can also cause problems later if further work on the product is required.

\subsubsection{Product Lines and Portfolios}

The aim of the above paragraphs was to distinguish the concept of project from product, but it is not really as simple as that. There are many potential layers of complexity above the simple product. The essential distinction between product and project remains, but development projects may end up affecting more than one product at a time, or working on components which are shared between different products.

One such additional layer is the idea of a product line. A software product line consists of several products which share similarities (and often key parts of their software or hardware) but have been adapted to suit different needs or requirements. a Software manufacturer may, for example, release `Home', `Pro' and `Enterprise' versions of the same core product, or provide high-value customers with variants including the customer's own branding and specialist features.

Similar to a product line, a software portfolio is also a collection of software products, but these typically have less in common than the products in a product line. For example, a manufacturer may have a product portfolio containing a word processor, a spreadsheet and a slide presentation application. In such scenarios there is often some common code shared between the products, but not always. Software developers may stick with one product, or work on different projects on different products in the portfolio.

