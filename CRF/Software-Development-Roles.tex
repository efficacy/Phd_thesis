\subsubsection{Client, Customer, or Product Owner}
This role represents the person or organisation which requires or sponsors a product or project. In some cases there is a specific client or customer, and in others there is only a general idea of the characteristics or needs of potential or current customers. Where there is no specific customer, this role is sometimes filled by a customer representative such as a \emph{Product Owner}. The job of this role is to arbitrate on choices and make decisions which affect the overall suitability of the product or project. It is important to note that a \emph{Customer} in this sense is not necessarily the same as a \emph{User}. Although it is tempting to assume that features and product behaviour are selected for the benefit of the end user, in commercial software contexts what matters is the opinion of whoever is sponsoring or paying for the product development.

\subsubsection{Project Manager}
The \emph{Project Manager} role is responsible for ensuring that a project completes on time and on budget, and meets the aims for which it was initiated. In cases where these three things cannot all be achieved, the Project Manager is responsible for managing priorities.

\subsubsection{Architect}
The \emph{Architect} role is responsible for making big technical choices which determine the direction of a product and the projects associated with it. Often this involves tasks such as dividing a large system into smaller subsystems with specific responsibilities and selecting major infrastructure components such as operating systems, databases and cloud platforms.

\subsubsection{Designer}
The \emph{Designer} role has different connotations for different organisations and products.  The distinction between this role and the \emph{Programmer} role, which often also includes elements of design, is that the pure Designer role does not usually do any programming as such.

\subsubsection{Programmer}
The \emph{Programmer} role is responsible for creating, updating, and fixing the software parts of a project. Historically this role was sometimes split into separate \emph{Analyst} and \emph{Programmer} roles, with an Analyst designing the algorithms and flow of the software for a Programmer to enter. This distinction has largely disappeared, and the Programmer (or often \emph{Developer}) role being responsible for design, coding and some testing of specific parts of a system.

\subsubsection{Tester}
The \emph{Tester} role is responsible for ensuring that the system works as intended. This may happen at the end of the development process, or spread throughout. Testers may test manually by using a system as if they are a real user, or they may be responsible for constructing and running automated tests. Some testers may have the final say on whether a product, or a particular version of a product, is released at all.

\subsubsection{Build, Configuration Management and Deployment}
For the very simple software often seen in teaching environments, this role may be virtually nonexistent, with software only run on one machine and `deployed' by saving or copying a single file. In more complex products, this role can become a lot more involved, particularly when product lines or product portfolios are involved, or when software needs to be deployed to multiple collaborating systems.

\subsubsection{Operation and Maintenance}
The \emph{Operation and Maintenance} role is particularly important in web applications and \emph{Software as a Service} (SAAS) products. The role is responsible for ensuring that the server-side parts of the application are available and functioning correctly for end users and client-side code to use. This can involve jobs such as monitoring availability and managing responsiveness, as well as increasing or reducing the numbers of participating machines. Vitally, this role has also become responsible for backups and resilience, attack prevention and mitigation, and other cybersecurity functions.

\subsubsection{Documentation and Support}
The final role in this short list is that of \emph{Documentation and Support}. No software product exists in a vacuum, and there will always be a need for some sort of documentation and some sort of support to enable people to use the system. This can be a simple as a `README' file, or as complex as a multi-layered 24-hours-a-day, 7-days-a-week, support structure backed up by websites, books, videos and other media.

